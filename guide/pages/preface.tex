% !TEX root = ../guide.tex
% introduction.tex
%
% Copyright 2016 Zheng Xie <xie.zheng777@gmail.com>
% https://github.com/Tedxz/xjtuthesis-x
%
% This work may be distributed and/or modified under the
% conditions of the LaTeX Project Public License, either version 1.3
% of this license or (at your option) any later version.
% The latest version of this license is in
%   http://www.latex-project.org/lppl.txt
% and version 1.3 or later is part of all distributions of LaTeX
% version 2005/12/01 or later.
%
% This work has the LPPL maintenance status `maintained'.
%
% The Current Maintainer of this work is Zheng Xie.
%
% xjtuthesis-x is a Derived Work of xjtuthesis. The original maintainer of
% xjtuthesis is Weisi Dai (multiple1902 <multiple1902@gmail.com>),
% who published the project on https://code.google.com/p/xjtuthesis/ (no
% longer accessable). Currently, xjtuthesis is maintained by Aetf, and can
% be accessed on https://github.com/Aetf/xjtuthesis.
%
% xjtuthesis-x includes bug fixes, new features and a user guide.
% For detail, please refer to Readme.md.
%
% If you want to contribute to xjtuthesis-x or become the maintainer of
% xjtuthesis-x, please feel free to contact me.

\chapter{前言}
\echapter{Preface}

维护\texttt{xjtuthesis-x}的最初目的是为了个人使用。数月前笔者决定使用\LaTeX 完成毕业设计论文,并从数个现存的西安交通大学的\LaTeX 模板中选择了\texttt{xjtuthesis}使用。在实际的写作过程中,遇到了许许多多的问题,其中有些是因为笔者对\LaTeX 的使用不熟练,另外一些则是因为\texttt{xjtuthesis}存在些许bug和功能上的缺陷。起初,笔者按照《西安交通大学本科毕业设计(论文)工作手册》中的规范修复了少数bug,并向\texttt{xjtuthesis}提交了相关代码。随着后期自己加入了新的功能(如伪代码框)和一些《工作手册》中未作规定的调整(如脚注的行距),考虑到这些改动可能不适合合并到原\texttt{xjtuthesis}项目,便自行维护了一个单独的分支,即\texttt{xjtuthesis-x}。写作过程中为了实现各种功能和修复各种不如意的排版问题,笔者常常边在Stack Exchange上查阅一边实验到天亮;心想如果能有一个详尽的模板使用手册则会节省很多时间。\texttt{xjtuthesis}中包含了一份许佳贡献的样例,覆盖了一些常用功能,是非常好的参考;但是对于模板功能的介绍不够详细,也缺少了一些特殊情况的应对技巧。为了方便将来使用本模板的同学,笔者以\LaTeX 新手的视角写作了这份使用手册,望能对将来用\LaTeX 完成毕业论文的学弟学妹起到一点帮助。限于个人经历,本文所涉及到的内容以西安交通大学本科生的毕业论文为出发点;硕士和博士论文能否原样参考则需要读者对照相应的论文规范自行斟酌。

本文一方面作为\texttt{xjtuthesis-x}的用户文档,不介绍\LaTeX 的基本知识,而是告诉读者模板提供了哪些需要了解的功能,应该如何规范地应用。虽然熟练的\LaTeX 用户可能知晓相同功能的不同实现方法,但是不使用现成方法不仅增加了麻烦,而且可能还会引入潜在的不兼容问题。对于\LaTeX 的入门,网络上有很多的现成资料。
另一方面,本文还旨在帮助同学们解决论文写作过程中的各种问题。因此,本文还掺杂了少许笔者的论文写作心得,和论文提交过程中需要用到的\LaTeX 技巧。由于\texttt{xjtuthesis-x}是\texttt{xjtuthesis}的衍生工作,本文中的许多内容也适用于\texttt{xjtuthesis}的用户。

在内容开始之前,必须要感谢原项目\texttt{xjtuthesis}的发起者Weisi Dai学长,现在的维护者Aetf,以及其他的许多贡献者。多亏他们,西安交通大学的许多同学才能使用\LaTeX 完成他们的毕业论文。\texttt{xjtuthesis}最初发布在Google Code\footnote{\url{https://code.google.com/p/xjtuthesis/}},现在\texttt{xjtuthesis}可以在Github获取\footnote{\url{https://github.com/Aetf/xjtuthesis}}。\texttt{xjtuthesis-x}\footnote{\url{https://github.com/Tedxz/xjtuthesis-x}}作为其衍生工作,也遵循LPPL\footnote{LaTeX Project Public License,\url{http://latex-project.org/lppl/}}协议。如果希望对\texttt{xjtuthesis-x}贡献代码,可以在Github上提出Pull Request。由于笔者毕业后将不会花费大量时间继续修改该项目,也欢迎大家接替成为项目的维护者。
