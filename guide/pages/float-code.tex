% !TEX root = ../guide.tex
% float-figure.tex
%
% Copyright 2016 Zheng Xie <xie.zheng777@gmail.com>
% https://github.com/Tedxz/xjtuthesis-x
%
% This work may be distributed and/or modified under the
% conditions of the LaTeX Project Public License, either version 1.3
% of this license or (at your option) any later version.
% The latest version of this license is in
%   http://www.latex-project.org/lppl.txt
% and version 1.3 or later is part of all distributions of LaTeX
% version 2005/12/01 or later.
%
% This work has the LPPL maintenance status `maintained'.
%
% The Current Maintainer of this work is Zheng Xie.
%
% xjtuthesis-x is a Derived Work of xjtuthesis. The original maintainer of
% xjtuthesis is Weisi Dai (multiple1902 <multiple1902@gmail.com>),
% who published the project on https://code.google.com/p/xjtuthesis/ (no
% longer accessable). Currently, xjtuthesis is maintained by Aetf, and can
% be accessed on https://github.com/Aetf/xjtuthesis.
%
% xjtuthesis-x includes bug fixes, new features and a user guide.
% For detail, please refer to Readme.md.
%
% If you want to contribute to xjtuthesis-x or become the maintainer of
% xjtuthesis-x, please feel free to contact me.

\section{插入伪代码与程序段}
对于计算机相关专业的同学来说,在论文中插入伪代码或程序段是很常见的需求,故笔者在\texttt{xjtuthesis-x}中加入了相关的功能。在笔者看来,绝大多数情况下,在论文中插入代码都是以增加篇幅为目的,牺牲了文章的可读性和逻辑性的错误之举。在论文中,如要描述程序的相关内容,笔者一直遵从以下几个原则:
\begin{enumerate}
  \item 代码实现了算法时,用伪代码或流程图叙述;
  \item 代码实现了复杂度数学运算时,用公式叙述;
  \item 描述程序中类的设计时,用UML图,而不是插入类的定义或声明代码;
  \item 描述整个项目的结构时,用架构图,而不是整合各个模块的顶层代码;
  \item 仅在描述编程语言相关的实现细节时,插入小段代码,一般不得跨页。
\end{enumerate}
这些规则可以使论文更加规范,具有可读性,供读者参考。


\subsection{伪代码及可跨页伪代码}
伪代码是笔者在\texttt{xjtuthesis-x}中新增的功能。由于《工作手册》中没有关于代码和伪代码的格式规范,笔者在最初的版本上结合老师和同学的意见进行了多次修改,确定了题注在上,类似表格三线分割的浮动格式。算法~\ref{alg:kmeans}中是一段伪代码的示例。其对应的代码如图~\ref{alg:kmeans}所示。将示例代码中的\texttt{algorithm}环境改为\texttt{breakablealgorithm}环境,可以将伪代码块改成可以跨页的非浮动样式。这些功能基于\texttt{algorithm}和\texttt{algpseudocode}两个宏包实现,模板对相关的环境进行了修改以配合论文的样式。用户可以自行查阅这两个宏包的文档,学习相关用法或自行修改伪代码的样式。

目前,伪代码插入尚为一个不完美的功能。在插入跨页伪代码时,模板并不能根据其相对页面的位置自动移动/添加/去掉分割线,如最后一行代码恰逢页末,可能会遇到最后一条分割线被插入至下一页起始的情况。另外,当非跨页代码段位于页面顶端时,其位置可能会有向下的微小偏移\footnote{可以通过使用\texttt{vskip}语句临时调整。浮动元素位置的调整涉及\texttt{intextsep}、\texttt{floatsep}、\texttt{textfloatsep}三个属性的调整,并牵扯到\LaTeX 的排版机制(参考\url{http://tex.stackexchange.com/questions/32614/vertical-spacing-in-floats})。当加入题注时,问题则更加复杂。当前的模板优先保证图片和表格的正确排版,取了折衷方案。}。所幸这些问题可以通过微调论文内容避免,读者还是可以权衡使用两种不同的伪代码环境。如果有这些问题的解决方案,欢迎向\texttt{xjtuthesis-x}贡献代码。

\begin{algorithm}[h]
  \caption{K-means算法}
  \label{alg:kmeans}
  \begin{tabular}{ll}
    \textbf{Input:}   & $X=\{\bm{x}_{1\dots n}\}$: dataset \\
                      & $k$: number of clusters \\
    \textbf{Output:}  & $C$: set of centroids \\
                      & $G_{1 \ldots k}$: the set of clusters \\
  \end{tabular}
  \algomidrule
  \begin{algorithmic}[1]
    \Procedure {K-Means}{$X$, $k$}
      \State \textbf{initialize} $C=\{\bm{c}_{1 \dots k}\}$ at random
      \Repeat
        \State $G_{i \dots k} \leftarrow \varnothing$
        \For {$\forall \bm{x} \in X$}
          \State $i \leftarrow \argmin\limits_i distance(\bm{x}, \bm{c}_i)$
          \State $G_i \leftarrow G_i \cup \{\bm{x}\}$
          \Comment $G_i$: the i-th cluster
        \EndFor
        \State $\bm{c}_i \leftarrow \frac{1}{|G_i|}\sum\limits_{\bm{x} \in G_i} {\bm{x}}$
      \Until no centroid moved
      \State \textbf{return} $C$, $G_{1 \ldots k}$
    \EndProcedure
  \end{algorithmic}
\end{algorithm}

\begin{figure}[h]
  {
  \setstretch{1.2}%
  \fontsize{10pt}{12pt}\selectfont
  \setmainfont{Courier New}
  \begin{lstlisting}[showstringspaces=false,numbers=left,xleftmargin=3em]
\begin{algorithm}
  \caption{K-means算法}
  \label{alg:kmeans}
  \begin{tabular}{ll}
    \textbf{Input:}   & $X=\{\bm{x}_{1\dots n}\}$: dataset \\
                      & $k$: number of clusters \\
    \textbf{Output:}  & $C$: set of centroids \\
                      & $G_{1 \ldots k}$: the set of clusters \\
  \end{tabular}
  \algomidrule
  \begin{algorithmic}[1]
    \Procedure {K-Means}{$X$, $k$}
      \State \textbf{initialize} $C=\{\bm{c}_{1 \dots k}\}$ at random
      \Repeat
        \State $G_{i \dots k} \leftarrow \varnothing$
        \For {$\forall \bm{x} \in X$}
          \State $i\leftarrow\argmin\limits_i distance(\bm{x}, \bm{c}_i)$
          \State $G_i \leftarrow G_i \cup \{\bm{x}\}$
          \Comment $G_i$: the i-th cluster
        \EndFor
        \State $\bm{c}_i \leftarrow \frac{1}{|G_i|}
               \sum\limits_{\bm{x} \in G_i} {\bm{x}}$
      \Until no centroid moved
      \State \textbf{return} $C$, $G_{1 \ldots k}$
    \EndProcedure
  \end{algorithmic}
\end{algorithm}

  \end{lstlisting}
  }
\caption{算法~\ref{alg:kmeans}对应的代码}
\label{fig:kmeans}
\end{figure}

\subsection{程序段}
基于本章开头的所述,笔者并不赞成无节制地向论文中插入程序段的行为。为此,\texttt{xjtuthesis-x}并没有确定一种规范的代码插入的格式。但对于计算机的学生而言,插入程序段可能是不可避免的。对于要插入程序段的读者,可以尝试遵从以下规则,以提高文章的可读性:
\begin{enumerate}
  \item 代码符合规范。包括合理地缩进、变量命名、断行,清晰的注释等。
  \item 使用等宽字体。如Courier New或Consolas等。
  \item 标注行号。
  \item 每段代码的长度不要超过一页。
\end{enumerate}

插入代码段使用\texttt{listings}宏包实现,已在模板中包含。如同本文中的许多代码段,笔者常将\texttt{lstlistings}环境嵌入在\texttt{figure}环境中作为图片插入,插入方法如图~\ref{fig:listings}所示。其中的参数可以根据实际情况调节,可以查阅\texttt{listings}宏包的文档以便更加精确地调整格式。如果论文中需要插入的代码量比较多,也可以自行创建一个标号,如``程序''。


\begin{figure}[h]
  {
  % the following line is a trick that make you able to insert "\end{lstlisting}" in the code.
  \lstnewenvironment{TeXlstlisting}{\lstset{showstringspaces=false,numbers=left,xleftmargin=3em}}{}
  \setstretch{1.2}%
  \fontsize{10pt}{12pt}\selectfont
  \setmainfont{Courier New}
  \begin{TeXlstlisting}
\begin{figure}[h] {
  \setstretch{1.2}%
  \fontsize{10pt}{12pt}\selectfont
  \setmainfont{Courier New}
  \begin{lstlisting}[language=tex,showstringspaces=false,numbers=left]

  % 此处插入代码

  \end{lstlisting}
}
\caption{代码题注}
\label{fig:listings}
\end{figure}

  \end{TeXlstlisting}
  }
\caption{代码插入样例}
\label{fig:listings}
\end{figure}
